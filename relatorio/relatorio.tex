\documentclass[12pt]{article}

% Fonte com acentos
\usepackage[brazil]{babel} \usepackage[utf8]{inputenc} \usepackage[T1]{fontenc}
\usepackage[lighttt]{lmodern} \usepackage[top=1.4in, bottom=1in, left=1in,
right=1in]{geometry} \usepackage{graphicx} % para adicionar imagem
\usepackage{listings} \usepackage{amsmath} \usepackage{indentfirst}

\begin{document}

\setlength{\parskip}{0.2cm}

\title{Análise de código e eficiência do método do Gradiente}

\author{Aryane Ast dos Santos\\ Kevin Katzer}

\date{23 de novembro de 2014}

\maketitle

\tableofcontents

\section{Introdução}
Motivação...

\section{Verificação de uso de memória com Valgrind}\label{sec:Valgrind}

Ao executar a ferramenta Valgrind para se obter informações sobre vazamento de
memória no programa gradSolver, foi possível observar 5 erros, todos em
contextos diferentes, além de 16 allocações e apenas 2 liberações de memória.

Os resultados da execução do programa são parcialmente apresentados na
figura~\ref{fig:valgrindOut}.

\begin{figure}[htb]
\begin{tt}\noindent
==29599== Command: ./gradSolver -r 5\\
==28949== HEAP SUMMARY:\\
==28949==     in use at exit: 560 bytes in 14 blocks\\
==28949==   total heap usage: 16 allocs, 2 frees, 800 bytes allocated\\
==28949== LEAK SUMMARY:\\
==28949==    definitely lost: 560 bytes in 14 blocks\\
==28949== ERROR SUMMARY: 5 errors from 5 contexts (suppressed: 0 from 0)
\end{tt}\caption{Saída do Valgrind}\label{fig:valgrindOut}
\end{figure}

Aqui o gradSolver foi executado com uma matriz quadrada de ordem 5, porém os
mesmos problemas listados na figura~\ref{fig:valgrindOut} são encontrados em
execuções de matrizes de qualquer dimensão. E de maneira análoga, ao resolver os
problemas apresentados, numa execução com matriz maior, eles ficam também
automaticamente resolvidos.

Para contornar os vazamentos de memória encontrados, foi necessário liberar a
memória dos vetores alocados explicitamente como o vetor x na função main, o
vetor aux em calcGrad e o vetor r de resíduo na função gradSolver. Além disso,
no main, foram adicionados frees para os ponteiros para char das flags do
getopt.

\section{Arquitetura do computador}\label{sec:likwid}

Utilizando a ferramenta likwid-topology, é possível obter as seguintes
informações sobre a arquitetura do computador utilizado para os testes de
performance.

\begin{figure}[ht]
\begin{tt}\noindent
CPU type:   AMD Interlagos processor \\
\\
Hardware Thread Topology\\
Sockets:    1 \\
Cores per socket:   6 \\
Threads per core:   1 \\
HWThread    Thread      Core        Socket\\
0       0       0       0\\
1       0       1       0\\
2       0       2       0\\
3       0       3       0\\
4       0       4       0\\
5       0       5       0\\
Socket 0: ( 0 1 2 3 4 5 )\\
\\
Cache Topology\\
Level:  1\\
Size:   16 kB\\
Cache groups:   ( 0 ) ( 1 ) ( 2 ) ( 3 ) ( 4 ) ( 5 )\\
Level:  2\\
Size:   2 MB\\
Cache groups:   ( 0 1 ) ( 2 3 ) ( 4 5 )\\
Level:  3\\
Size:   8 MB\\
Cache groups:   ( 0 1 2 3 4 5 )\\
\\
NUMA Topology\\
NUMA domains: 1 \\
Domain 0:\\
Processors:  0 1 2 3 4 5\\
Relative distance to nodes:  10\\
Memory: 152.301 MB free of total 3678.97 MB\\
\end{tt}\caption{Saída do likwid-topology}\label{fig:topologyOut}
\end{figure}

Como podemos notar na figura~\ref{fig:topologyOut} ...

\section{Análise do cálculo do fator lambda}\label{sec:lambda}
\subsection{flop operations}
\subsection{mem utilization}
\subsection{explicar graficos flops\_dp, cache, mem}
\subsection{melhoria}

\section{Análise do cálculo do resíduo}\label{sec:residuo}
\subsection{flop operations}
\subsection{mem utilization}
\subsection{explicar graficos flops\_dp, cache, mem}
\subsection{melhoria}

\end{document}
